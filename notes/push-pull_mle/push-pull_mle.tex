\pdfoutput=1
% ***********************************************************
% ******************* PHYSICS HEADER ************************
% ***********************************************************
% Version 2
%\documentclass[11pt,twocolumn]{article}
\documentclass[aps,onecolumn,superscriptaddress,notitlepage]{revtex4-1}


\usepackage{amsmath} % AMS Math Package
\usepackage{amsthm} % Theorem Formatting
\usepackage{amssymb}	% Math symbols such as \mathbb
\usepackage{physics}
\usepackage[pdftex]{graphicx}
\usepackage{hyperref} %clickable references.
\hypersetup{
    colorlinks,
    citecolor=blue,
    filecolor=black,
    linkcolor=red,
    urlcolor=blue
}
\usepackage{xcolor}
\usepackage{url}

\newcommand{\comment}[1]{{\color{blue}#1}}
\newcommand{\edit}[1]{{\color{violet}#1}}
\newcommand{\del}[1]{{\color{red}\st{#1}}}



% ***********************************************************
% ********************** END HEADER *************************
% ***********************************************************

\begin{document}

\title{Maximum Likelihood Estimation with Push-Pull Noise Models}
\author{Jason W. Rocks and Pankaj Mehta}
\author{Pankaj Mehta}
\maketitle

In this set of notes, we describe the maximum likelihood estimation techniques used to fit our push-pull amplifier models to experimental data.
 
\begin{align}
P(\log(GFP)) &= \frac{1}{\sqrt{2\pi\sigma_{\mathrm{GFP}}^2 }}\exp\qty(-\frac{(\log(GFP)-\mu_{GRP})^2}{2\sigma_{\mathrm{GFP}}^2})
\end{align}

\begin{align}
P(\log(anti)) &= \frac{1}{\sqrt{2\pi\sigma_{\mathrm{anti}}^2 }}\exp\qty(-\frac{(\log(anti)-\mu_{anti})^2}{2\sigma_{\mathrm{anti}}^2})
\end{align}


\begin{gather}
P(\log(anti) \qand \log(GFP)) = \frac{1}{\sqrt{2\pi \qty(\sigma_{\mathrm{anti}}^2\sigma_{\mathrm{GFP}}^2 - \sigma_{\mathrm{anti}, \mathrm{GFP}}^2)}}\\
\times\exp\qty(\frac{\sigma_{\mathrm{GFP}}^2(\log(anti)-\mu_{anti})^2 + \sigma_{\mathrm{anti}}^2 (\log(GFP)-\mu_{GFP})^2 - 2\sigma_{\mathrm{anti}, \mathrm{GFP}}(\log(anti)-\mu_{anti})(\log(GFP)-\mu_{GFP})}{2\qty(\sigma_{\mathrm{anti}}^2\sigma_{\mathrm{GFP}}^2 - \sigma_{\mathrm{anti}, \mathrm{GFP}}^2)})
\end{gather}

\begin{align}
P(\log(anti)| \log(GFP)) &= \frac{P(\log(anti) \qand \log(GFP)) }{P(\log(GFP))}\\
&= \sqrt{\frac{\sigma_{\mathrm{GFP}}^2}{\qty(\sigma_{\mathrm{anti}}^2\sigma_{\mathrm{GFP}}^2 - \sigma_{\mathrm{anti}, \mathrm{GFP}}^2)}}  \exp\qty(-\frac{\qty(\sigma_{\mathrm{GFP}}^2(\log(anti)-\mu_{anti}) - \sigma_{\mathrm{anti}, \mathrm{GFP}} (\log(GFP)-\mu_{GFP}))^2}{2\sigma_{\mathrm{GFP}}^2\qty(\sigma_{\mathrm{anti}}^2\sigma_{\mathrm{GFP}}^2 - \sigma_{\mathrm{anti}, \mathrm{GFP}}^2)})\\
&= \frac{1}{\sqrt{\Sigma^2}}\exp\qty(-\frac{\qty(\log(anti)-A\log(GFP)-B)^2}{2\Sigma^2})
\end{align}

\begin{align}
\mathcal{L} &= -\frac{1}{N}\sum_i\log\qty(P(\log(anti_i)| \log(GFP_i)))\\
 &=  \frac{1}{2\Sigma^2N}\sum_i\qty(\log(anti_i)-A\log(GFP_i)-B)^2 +\frac{1}{2}\log(\Sigma^2)
\end{align}



\end{document}

