\pdfoutput=1
% ***********************************************************
% ******************* PHYSICS HEADER ************************
% ***********************************************************
% Version 2
%\documentclass[conference, onecolumn]{IEEEtran}
\documentclass[aps,onecolumn,superscriptaddress, notitlepage]{revtex4-1}


\usepackage{amsmath} % AMS Math Package
%\interdisplaylinepenalty=2500
\usepackage{amsthm} % Theorem Formatting
\usepackage{amssymb}	% Math symbols such as \mathbb
\usepackage{physics}
%\usepackage{cite}
\usepackage[pdftex]{graphicx}
\usepackage{xcolor}

%\hyphenation{op-tical net-works semi-conduc-tor}



\newcommand{\comment}[1]{{\color{blue}#1}}
\newcommand{\edit}[1]{{\color{magenta}#1}}

% ***********************************************************
% ********************** END HEADER *************************
% ***********************************************************

\begin{document}
\title{Receptor Thermodynamic to Kinematic Conversion}

%\author{\IEEEauthorblockN{Jason W. Rocks and Pankaj Mehta}
%\IEEEauthorblockA{Department of Physics, Boston University\\
%Boston, Massachusetts 02215\\
%Email:}
\author{Jason W. Rocks}
\maketitle


We have the following concentrations for the three constituent elements:
\begin{itemize}
\item Receptor domain 1: $[R_1]$
\item Receptor domain 2: $[R_2]$
\item Ligand: $[L]$
\end{itemize}

Now we have the following configurational states:
\begin{center}
\begin{tabular}{ |c|c|c| } 
 \hline
 State/Concentration & Configuration & Energy \\\hline 
 $[R_1]$ & Unbound receptor  & $\epsilon_0$ \\ 
 $[C_{off}] = [R_1-R_2]$ & Bound receptor w/o ligand & $\epsilon_0+\Delta\epsilon_b-k_bT\ln [R_2]$  \\ 
 $[C_{on}] = [R_1-R_2-L]$ & Bound receptor w/ ligand  & $\epsilon_0+\Delta\epsilon_b-k_bT\ln [R_2] + \Delta\epsilon_L-k_bT\ln [L]$ \\
 \hline
\end{tabular}
\end{center}


The probability of the unbound state is
\begin{align}
p(R_1) &= \frac{[R_1]}{[R_1]+[R_1-R_2]+[R_1-R_2-L]} = \frac{e^{-\beta\epsilon_0}}{e^{-\beta\epsilon_0} + e^{-\beta\qty(\epsilon_0+\Delta\epsilon_b-k_bT\ln [R_2])} +  e^{-\beta\qty(\epsilon_0+\Delta\epsilon_b-k_bT\ln [R_2] + \Delta\epsilon_L-k_bT\ln [L])}}\\
&= \frac{1}{1 +   [R_2]e^{-\beta\Delta\epsilon_b}  +  [R_2][L]e^{-\beta(\Delta\epsilon_B+\Delta\epsilon_L)} }
\end{align}
Similarly, the probabilities of the other two states are 

\begin{align}
p(C_{off}) &=  \frac{[R_1-R_2]}{[R_1]+[R_1-R_2]+[R_1-R_2-L]} =  \frac{[R_2]e^{-\beta\Delta\epsilon_b}}{1 +   [R_2]e^{-\beta\Delta\epsilon_b}  +  [R_2][L]e^{-\beta(\Delta\epsilon_B+\Delta\epsilon_L)}}\\
p(C_{on}) &=  \frac{[R_1-R_2-L]}{[R_1]+[R_1-R_2]+[R_1-R_2-L]} =  \frac{ [R_2][L]e^{-\beta(\Delta\epsilon_B+\Delta\epsilon_L)}}{1 +   [R_2]e^{-\beta\Delta\epsilon_b}  +  [R_2][L]e^{-\beta(\Delta\epsilon_B+\Delta\epsilon_L)}}\\
\end{align}


Now we can convert to kinetic coefficients. There are three transitions we can consider, each related to a ratio of probabilities.
First the transitions from unbound to off
\begin{align}
k^+_{R_1\rightarrow off} [R_1][R_2] &= k^-_{R_1\rightarrow off}[C_{off}]
\end{align}
From this, we see that the kinetic coefficients are given by,
\begin{align}
\frac{k^+_{R_1\rightarrow off}}{k^-_{R_1\rightarrow off}} &= \frac{[C_{off}]}{[R_1][R_2]} = \frac{p(C_{off})}{p(R_1)[R_2]} =   e^{-\beta\Delta\epsilon_b}
\end{align}

Next, the transition from off to on,
\begin{align}
k^+_{off\rightarrow on}[C_{off}][L] = k^-_{off\rightarrow on}[C_{on}]
\end{align}
which gives us
\begin{align}
\frac{k^+_{off\rightarrow on}}{k^-_{off\rightarrow on}} &= \frac{[C_{on}]}{[C_{off}][L]} = \frac{p(C_{on})}{p(C_{off})[L]} = e^{-\beta\Delta\epsilon_L}
\end{align}

Finally, the transition from on to unbound,
\begin{align}
k^+_{on\rightarrow R_1}[C_{on}] = k^-_{on\rightarrow R_1}[R_1][R_2][L]
\end{align}
which gives us
\begin{align}
\frac{k^+_{on\rightarrow R_1}}{k^-_{on\rightarrow R_1}} = \frac{[R_1][R_2][L]}{[C_{on}]} = \frac{p(R_1)[R_2][L]}{p(C_{on})} = \frac{1}{e^{-\beta(\Delta\epsilon_B+\Delta\epsilon_L)}}\label{eq:kin3}
\end{align}

Note that the kinetic coefficients are related such that
\begin{align}
\frac{k^+_{R_1\rightarrow off}}{k^-_{R_1\rightarrow off}}\frac{k^+_{off\rightarrow on}}{k^-_{off\rightarrow on}}\frac{k^+_{on\rightarrow R_1}}{k^-_{on\rightarrow R_1}} = 1
\end{align}


Finally, we want the combined concentration of $[C_{off}]$ and  $[C_{on}]$,
\begin{align}
\frac{[C_{tot}]}{[R_1]+[C_{off}]+[C_{on}]} &= p(C_{off}) + p(C_{on})\\
&= \frac{e^{-\beta\Delta\epsilon_b} + [R_2][L]e^{-\beta(\Delta\epsilon_B+\Delta\epsilon_L)}}{1 +   [R_2]e^{-\beta\Delta\epsilon_b}  +  [R_2][L]e^{-\beta(\Delta\epsilon_B+\Delta\epsilon_L)}}\\
&= p(R_1) [R_2]e^{-\beta\Delta\epsilon_b}\qty(1 + [L]e^{-\beta\Delta\epsilon_L})\\
&= \frac{[R_1]}{[R_1]+[C_{off}]+[C_{on}]} [R_2]\frac{k^+_{R_1\rightarrow off}}{k^-_{R_1\rightarrow off}}\qty(1+ [L]\frac{k^+_{off\rightarrow on}}{k^-_{off\rightarrow on}})
\end{align}

Multiplying both sides by the total concentration,
\begin{align}
[C_{tot}] &= [R_1][R_2]\frac{k^+_{R_1\rightarrow off}}{k^-_{R_1\rightarrow off}}\qty(1+ [L]\frac{k^+_{off\rightarrow on}}{k^-_{off\rightarrow on}})
\end{align}


\end{document}