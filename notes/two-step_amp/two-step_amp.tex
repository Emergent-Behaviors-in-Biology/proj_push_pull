%% BASIC CLASS FILE
%\documentclass[prl,twocolumn,amsmath]{revtex4-1}
\documentclass[preprint,onecolumn,amsmath]{revtex4-1}

%% ADDITIONAL OPTIONAL STYLE FILES
\usepackage{graphicx}
\usepackage{amssymb,amsfonts,amsmath, dsfont}

%% OPTIONAL MACRO DEFINITIONS
\def\s{\sigma}

%  shorter names for greek alphabets
\def \g{\gamma}    \def \a{\alpha}
\def \w{\omega}    \def \b{\beta} 
\def \s{\sigma}    \def \k{\kappa}  
\def \e{\epsilon}   \def \r{\rho} 
\def \th{\vec{\theta}}   \def \d{\delta} 
\def \k{\kappa}    \def \l{\lambda} 
\def \z{\zeta}      
 
\def \O{\Omega}   \def \S{\Sigma} 
\def \G{\Gamma}   \def \D{\Delta} 
\def \Lam{\Lambda} 
\def \h{\hbar}   %  \h won't be used for any greek letter 


% operators
\def \E{\rm{E}} % expectation
\def \Var{\rm{Var}} % variance
\def \Cov{\rm{Cov}} % covariance
 
\def \f{\frac} 
\def \del{\partial}    % for writing partial derivatives 
 
\def \hf{\tfrac{1}{2}} 
\def \HF{\dfrac{1}{2}}  % small and big half's. 
\def \HQ{\dfrac{1}{4}}
 
\def \ord{\mathcal{O}} 
\def \ra{\rightarrow} 
\def \>{\rangle} 
\def \<{\langle} 
\def\dg{^\dagger} 
\def\lba{\left(}    %don't know if these are already defined, have to check 
\def\rba{\right)} 
\def\lbc{\left[} 
\def\rbc{\right]} 
\def\lbb{\left\{} 
\def\rbb{\right\}} 
\def\bra{\langle} 
\def \ket{\rangle} 
\def\ord{\mathcal{O}} 
 
\def\CH{\cal{H}} 
 
\def\be{\begin{equation}} 
\def\ee{\end{equation}} 
\def\longrightharpoonup{\relbar\joinrel\rightharpoonup}
\def\longleftharpoondown{\leftharpoondown\joinrel\relbar}

\def\longrightleftharpoons{
  \mathop{
    \vcenter{
      \hbox{
      \ooalign{
        \raise1pt\hbox{$\longrightharpoonup\joinrel$}\crcr
	  \lower1pt\hbox{$\longleftharpoondown\joinrel$}
	  }
      }
    }
  }
}

\newcommand{\rates}[2]{\displaystyle
  \mathrel{\longrightleftharpoons^{#1\mathstrut}_{#2}}}
\newcommand \bea {\begin{eqnarray}} 
\newcommand \eea {\end{eqnarray}} 
\newcommand \sign {\hbox{sign}} 
\newcommand{\nn} {\nonumber}
\newcommand{\vect}[1]{\boldsymbol{#1}}

\DeclareMathOperator*{\argmax}{arg\,max}
\DeclareMathOperator{\sech}{sech}
\DeclareMathOperator\erf{erf}

\def \bvec{\textbf}
\def \mat{\text}


%%%%%%%%%%%%%%%%%%%%%%%%%%%%%%%%%%%%%%%%%%%%%%%%%%%%%%%%%%%%%%%

\begin{document}

\title{Two Step Amplifier System}

\author{Pankaj Mehta}
\author{Jason Rocks}
\affiliation{Dept. of Physics, Boston University, Boston, MA 02215}

\date{\today}

\begin{abstract}
These are notes on modeling the stop step amplifier system
\end{abstract}

\maketitle

Let us start with the basic set-up. We have a receptor $R$ which here we will assume refers to the dimerized, active form of the receptor. This
receptor can bind and phosphorylate a writer (i.e. kinase) $W$. The kinase $K$ can also bind an eraser (i.e phosphatase) $E$. Importantly, the $R$ and $P$ cannot
be bound to the kinase at the same time. When the K is phosphorylated, it can bind a substrate $S$ and phosphorylate $S$. The substrate can also bind
a second eraser $E_2$.

\section{Kinase Equations}
Let us start at viewing this from the view point of the phosphorylated kinase $W-P$. This can be found in one fo the following configurations
\begin{itemize}
\item $[W-P]_f$ -- concentration of unbound phophorylated kinase 
\item $[WR-P]$ -- concentration of  phophorylated kinase bound to active (dimerized) receptor  
\item $[WE-P]$ -- concentration of  phophorylated kinase bound to eraser  
\item $[WS-P]$ -- concentration of phophorylated kinase bound to substrate
\end{itemize}
We can also consider the states of the unphorphorylated kinase (note this cannot bind substrate as is SH3 domain)
\begin{itemize}
\item $[W]_f$ -- concentration of unbound phophorylated kinase 
\item $[WR]$ -- concentration of  phophorylated kinase bound to active (dimerized) receptor  
\item $[WE]$ -- concentration of  phophorylated kinase bound to active (dimerized) receptor  
\end{itemize}


Since all the interactions in each of the lists is thermodynamic, we can use thermodynamic models to write separate partition functions for each of these. Let
us denote the binding energies of the receptor to writer, phosphatase/substrate as $\Delta \epsilon_{WR}$ , $\Delta \epsilon_{WE}$, $\Delta \epsilon_{WS}$.
We then can write the probabilities of being in each of these states as (where the subscript $f$ depicts the concentration of free, unbound species)
\begin{align}
p_{[W-P]_f}^P &= {1 \over Z_{WP}} \\
p_{[WR-P]}^P &= {e^{-\beta (\Delta \epsilon_{WR} -k_B T \log{[R]_f}}  \over Z_{WP}}  \equiv { {[R]_f \over \alpha_{WR}} \over Z_{WP}} \\
p_{[WE-P]}^P &= {e^{-\beta (\Delta \epsilon_{WE} -k_B T \log{[E]_f})}  \over Z_{WP}}  \equiv { {[E]_f \over \alpha_{WE}} \over Z_{WP}} \\
p_{[SE-P]}^P &= {e^{-\beta (\Delta \epsilon_{SE} -k_B T \log{[S]_f})}  \over Z_{WP}}  \equiv { {[S]_f \over \alpha_{WE}} \over Z_{WP}} \\
Z_{WP} &= 1+ {[R]_f \over \alpha_{WR}}+{[E]_f \over \alpha_{WE}}+{[S]_f \over \alpha_{WE}}
\end{align}
Similarly, we can do this for unphosphorylated stuff and since binding energies (except to the substrate) don't depend on the phosphorylation state
we have
\begin{align}
p_{[W]_f} &= {1 \over Z_W} \\
p_{[WR]} &= {e^{-\beta (\Delta \epsilon_{WR} -k_B T \log{[R]_f}}  \over Z}  \equiv { {[R]_f \over \alpha_{WR}} \over Z_W} \\
p_{[WE]} &= {e^{-\beta (\Delta \epsilon_{WE} -k_B T \log{[E]_f})}  \over Z}  \equiv { {[E]_f \over \alpha_{WE}} \over Z_W} \\
Z_W &=1+ {[R]_f \over \alpha_{WR}}+{[E]_f \over \alpha_{WE}}
\end{align}

So now let us denote the total amount of writer (all, phosphorylated, and unphosphorylated) by  $[W_{tot}]$, $[W-P]$, and $[W]$ respectively.
Then, we know by definition we have that
\be
[W_{tot}] = [W-P] + [W].
\ee
Furthermore, the equations governing kinetics is just given by
\begin{align}
{d [W-P] \over dt} &= k_R^a [W] p_{[WR]}  + k_{bg}^a [W] - k_E^p [W-P]p_{[WE-P]}^P - k_{bg}^p [W-P] \\
&= k_R^a ([W_{tot}] - [W-P] )p_{[WR]}  + k_{bg}^a ([W_{tot}] -[W-P]) - k_E^p [W-P]p_{[WE-P]}^P - k_{bg}^p [W-P] 
\end{align}
At steady-state this gives
\be
{ [W-P] \over [W_{tot}] } ={ v_R^a p_{[WR]} + v_{bg}^a \over   v_R^a  p_{[WR]} + v_{bg}^a + v_E^p p_{[WE-P]}^P +1}
\ee
where we have defined the velocities $v_R^a = k_R^a/k_{bg}^p$, $v_{bg}^a = k_{bg}^a/k_{bg}^p$,$v_E^p = k_E^p/k_{bg}^p$.
This also gives combining equations above
\be
{ [W] \over [W_{tot}] }= 1- { [W-P] \over [W_{tot}] } = {v_E^p p_{[WE-P]}^P +1 \over  v_R^a  p_{[WR]} + v_{bg}^a + v_E^p p_{[WE-P]}^P +1}
\ee
Note that now we can use the probabilities above in terms of the free/unbound concentration of the receptor $[R_f]$, 
the free/unbound concentration of the eraser $[E_f]$, and free substrate concentration, and the total amount of kinase, $[W_{tot}]$.



\section{Receptor Equations}
Let us start with treating the receptor. To do this, we will just use the fact that dimerized 
receptor can be in three states, bound to the phosphorylated kinase, bound to unphosphorylated kinase. So we have the law
\be
[R_{tot}]= [R_f]+[WR]+[WR-P]
\ee
Furthermore, we can write the reader in three states and once again write the probability of these
\begin{align}
q_{[R]_f} &= {1 \over Z_{R}} \\
q_{[WR-P]} &= {e^{-\beta (\Delta \epsilon_{WR} -k_B T \log{[W-P]_f})}  \over Z_{R}} ={ {[W-P]_f \over \alpha_{WR}} \over Z_R}=
{ p_{[W-P]_f^P} {[W-P] \over \alpha_{WR}} \over Z_R} \\
q_{[WR]} &= {e^{-\beta (\Delta \epsilon_{WR} -k_B T \log{[W]_f})}  \over Z_{R}} ={ {[W]_f \over \alpha_{WR}} \over Z_R}=
{ p_{[W]_f} {[W] \over \alpha_{WR}} \over Z_R}\\
Z_R  &= 1+  {[W]_f \over \alpha_{WR}}+ {[W-P]_f \over \alpha_{WR}}=1+   p_{[W-P]_f}^P {[W-P] \over \alpha_{WR}}+ p_{[W]_f} {[W] \over
 \alpha_{WR}} 
\end{align} 
This of course gives us the equation
\be
{ [R_f] \over [R_{tot}]} = q_{[R]_f}={1 \over 1+   p_{[W]_f} {[W] \over \alpha_{WR}}+ p_{[W-P]_f}^P {[W-P] \over \alpha_{WR}}}
\ee

\section{Phosphatase Equation}

We can also write the phosphatase equations which are essentially identical. We have 
\be
[E_{tot}]= [E_f]+[WE]+[WE-P]
\ee
Furthermore, we can write the reader in three states and once again write the probability of these
\begin{align}
h_{[E]_f} &= {1 \over Z_{E}} \\
h_{[WE-P]} &= {e^{-\beta (\Delta \epsilon_{WE} -k_B T \log{[W-P]_f})}  \over Z_{E}} ={ {[W-P]_f \over \alpha_{WE}} \over Z_E}=
{ p_{[W-P]_f^P} {[W-P] \over \alpha_{WE}} \over Z_E} \\
h_{[WE]} &= {e^{-\beta (\Delta \epsilon_{WE} -k_B T \log{[W]_f})}  \over Z_{E}} ={ {[W]_f \over \alpha_{WE}} \over Z_E}=
{ p_{[W]_f} {[W] \over \alpha_{WE}} \over Z_R}\\
Z_R  &= 1+  {[W]_f \over \alpha_{WE}}+ {[W-P]_f \over \alpha_{WE}}=1+   p_{[W-P]_f}^P {[W-P] \over \alpha_{WE}}+ p_{[W]_f} {[W] \over
 \alpha_{WE}} 
\end{align} 
This of course gives us the equation
\be
{ [E_f] \over [E_{tot}]} = h_{[E]_f}={1 \over 1+   p_{[W]_f} {[W] \over \alpha_{WE}}+ p_{[W-P]_f}^P {[W-P] \over \alpha_{WE}}}
\ee


\section{Substrate Equation}
We now consider the substrate $[S]$ (we will not distinguish between phosphorylated or unphosphorylated substrate here). 
\begin{itemize}
\item $[S]_f$ -- concentration of unbound/free substrate 
\item $[WS-P]$ -- concentration of  substrate bound to phosphorylated writer
\item $[SE_2]$ -- concentration of substrate bound to it's own eraser/phosphatase ($E_2$).
\end{itemize}
As before we have that total number is conserved
\be
[S_{tot}]= [S]_f+[WS-P]+[SE_2-P]
\ee
Therefore, we can write usual partition function
\begin{align}
l_{[S]_f} &= {1 \over Z_{S}} \\
l_{[WS-P]} &= {e^{-\beta (\Delta \epsilon_{WS} -k_B T \log{[W-P]_f})}  \over Z_{S}} ={ {[W-P]_f \over \alpha_{WS}} \over Z_S}=
{ p_{[W-P]_f^P} {[W-P] \over \alpha_{WS}} \over Z_S} \\
l_{[SE2]} &= {e^{-\beta (\Delta \epsilon_{SE_2} -k_B T \log{[E2]_f})}  \over Z_{S}} ={ {[E2]_f \over \alpha_{SE2}} \over Z_S} 
={d_{[SE2]}{ [E2_{tot}] \over \alpha_{SE2}} \over Z_s} \\
 Z_S&= 1+  {[W-P]_f \over \alpha_{WE}}+ {[E2]_f \over \alpha_{SE2}}
 \end{align} 
Now we just need the usual kinetic equation
\begin{align}
{d [S-P] \over dt} &= k_W^a [W] l_{[WS-P]}  + k_{bg}^a [S] - k_{E2}^p [S-P]l_{[SE2]} - k_{bg}^p [S-P] \\
&= k_W^a ([S_{tot}] - [S-P] )l_{[WS-P]}  + k_{bg}^a ([S_{tot}] -[S-P]) - k_{E2}^p [S-P]l_{[SE2]}- k_{bg}^p [S-P] 
\end{align}
Setting this to zero gives ratio of phosphorylated substrate
\be
{[S-P] \over [S_{tot}]} ={ v_W^a l_{[WS-P]} +v_{bg}^a  \over v_W^a l_{[WS-P]}  + v_{E2}^p l_{[SE2]} + +v_{bg}^a+1}
\ee
and unphosphorylated substrate
\be
{[S] \over [S_{tot}]} ={ v_{E2}^p l_{[WS-P]} +1 \over v_W^a l_{[WS-P]}  + v_{E2}^p l_{[SE2]} + +v_{bg}^a+1}
\ee

\section{Second Phosphatase/Eraser}
The final equation we will need is the final phosphatase equation $[E2]$. This will essentially be the same derivation as above
\be
[E2_{tot}]= [E2_f]+[SE2]
\ee
Furthermore, we can write the reader in three states and once again write the probability of these
\begin{align}
d_{[E2]_f} &= {1 \over Z_{E2}} \\
d_{[SE2]} &= {e^{-\beta (\Delta \epsilon_{SE2} -k_B T \log{[S]_f})}  \over Z_{E2}}= {{[S_f] \over \alpha_{SE2}} \over Z_{E2}} \\
Z_{E2}  &= 1+{[S_f] \over \alpha_{SE2}}
\end{align} 
This of course gives us the equation
\be
{ [E2_f] \over [E2_{tot}]} = d_{[E2]_f} = {1 \over 1+{[S_f] \over \alpha_{SE2}}}
\ee



\end{document}

