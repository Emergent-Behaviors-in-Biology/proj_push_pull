\pdfoutput=1
% ***********************************************************
% ******************* PHYSICS HEADER ************************
% ***********************************************************
% Version 2
%\documentclass[11pt,twocolumn]{article}
\documentclass[aps,onecolumn,superscriptaddress,notitlepage]{revtex4-1}


\usepackage{amsmath} % AMS Math Package
\usepackage{amsthm} % Theorem Formatting
\usepackage{amssymb}	% Math symbols such as \mathbb
\usepackage{physics}
\usepackage[pdftex]{graphicx}
\usepackage{hyperref} %clickable references.
\hypersetup{
    colorlinks,
    citecolor=blue,
    filecolor=black,
    linkcolor=red,
    urlcolor=blue
}
\usepackage{xcolor}
\usepackage{url}

\newcommand{\comment}[1]{{\color{blue}#1}}
\newcommand{\edit}[1]{{\color{violet}#1}}
\newcommand{\del}[1]{{\color{red}\st{#1}}}



% ***********************************************************
% ********************** END HEADER *************************
% ***********************************************************

\begin{document}

\title{Quasi-thermodynamic Push-Pull Models}
\author{Jason W. Rocks}
\author{Pankaj Mehta}
\maketitle


\section{One-way Push Model}


In this model, we have a substrate $S^u$ and a writer (kinase) $W$. 
The writer is able to phosporylate the substrate from $S^u$ to $S^p$.
In this model, we do not explicitly include an eraser (phosphatase), 
but there is a dephosphoralation background rate (and a phosphoralation background rate).

\textbf{Reactions:}
\begin{gather}
W + S^u \leftrightarrows WS^u \rightarrow WS^p \leftrightarrows W + S^p
\end{gather}
\textbf{Background:}
\begin{gather}
S^p \rightarrow S^u\\
S^u \rightarrow S^p\\
WS^p \rightarrow WS^u\\
WS^u \rightarrow WS^p
\end{gather}


\textbf{Total Concentrations:}
We define total concentrations for each species and also separately for unphosporylated and phosporylated states.
\begin{itemize}
\item Total writer: $[W_T] = [W] + [WS^u] + [WS^p]$
\item Total substrate: $[S_T] = [S^u_T] + [S^p_T]$
\item Total unphosporylated substrate: $[S^u_T] = [S^u] + [WS^u]$
\item Total phosporylated substrate: $[S^p_T] = [S^p] + [WS^p]$
\end{itemize}
If we know the 

\textbf{Binding Energies:}
We assume the binding energies are the same for the unphosporylated and phosporylated states.
\begin{itemize}
\item Writer + substrate: $\Delta\epsilon_{WS}$
\end{itemize}

\textbf{One-way Reaction Rates:}
These are the reaction rates that cannot be described by binding energies.
\begin{itemize}
\item Phosphorylation of substrate by writer: $k_{WS}^p$
\item Background phosphorylation of substrate (independent of binding state): $k_S^p$
\item Background dephosphorylation of substrate (independent of binding state): $k_S^u$
\end{itemize}

\textbf{Partition Functions and Probabilities:}
We write down partition functions for each species based on the states listed in the total concentrations above. 
Partition function cannot mix unphosporylated and phosporylated states for the same species, 
so we do not write down a partition function for the total $S$.
We assume the unbound state of each species is zero energy (although this choice doesn't matter).
\begin{itemize}
\item Writer: $Z_W = 1 + e^{-\beta\qty(\Delta\epsilon_{WS} -k_BT \log[S^u])} + e^{-\beta(\Delta\epsilon_{WS} -k_BT \log[S^p])}$
\item Unphosphorylated substrate: $Z_{S^u} = 1 + e^{-\beta\qty(\Delta\epsilon_{WS} -k_BT \log[W])}$
\item Phosphorylated substrate: $Z_{S^p} = 1 + e^{-\beta\qty(\Delta\epsilon_{WS} -k_BT \log[W])}$
\end{itemize}
We definee the reaction velocity
\begin{align}
v_{WS} &= e^{-\beta\Delta\epsilon_{WS}}
\end{align}
The associated conditional probabilities are then
\begin{align}
p(W|S^u,S^p) &= \frac{[W]}{[W_T]} = \frac{1}{Z_W} =  \frac{1}{1 + v_{WS}([S^u] +[S^p])}\\
p(WS^u|S^u,S^p) &= \frac{[WS^u]}{[W_T]}  = \frac{e^{-\beta\qty(\Delta\epsilon_{WS} -k_BT \log[S^u])}}{Z_W} =  \frac{v_{WS}[S^u]}{1 + v_{WS}([S^u] +[S^p])}\\
p(WS^p|S^u,S^p) &= \frac{[WS^p]}{[W_T]}  = \frac{e^{-\beta\qty(\Delta\epsilon_{WS} -k_BT \log[S^p])}}{Z_W} =  \frac{v_{WS}[S^p]}{1 + v_{WS}([S^u] +[S^p])}
\end{align}
\begin{align}
p(S^u|W) &= \frac{[S^u]}{[S^u_T]} = \frac{1}{Z_{S^u}} = \frac{1}{1 + v_{WS}[W]}\\
p(WS^u|W) &= \frac{[WS^u]}{[S^u_T]} = \frac{ e^{-\beta\qty(\Delta\epsilon_{WS} -k_BT \log[W])}}{Z_{S^u}} = \frac{v_{WS}[W]}{1 + v_{WS}[W]}
\end{align}
\begin{align}
p(S^p|W) &= \frac{[S^p]}{[S^p_T]} = \frac{1}{Z_{S^p}} = \frac{1}{1 + v_{WS}[W]}\\
p(WS^p|W) &= \frac{[WS^p]}{[S^p_T]} = \frac{ e^{-\beta\qty(\Delta\epsilon_{WS} -k_BT \log[W])}}{Z_{S^p}} = \frac{v_{WS}[W]}{1 + v_{WS}[W]}
\end{align}


\textbf{Reactions in Detailed Balance:}
From the probabilities, we focus on the equations for the concentration of each species in its unbound state.
These correspond to the kinetic equations for reactions not involving (de)phosphorylation.
\begin{align}
\frac{[W]}{[W_T]} & = \frac{1}{1 + v_{WS}([S^u] +[S^p])}\label{eq:onewayW}\\
\frac{[S^u]}{[S^u_T]} &= \frac{1}{1 + v_{WS}[W]}\label{eq:onewaySu}\\
\frac{[S^p]}{[S^p_T]} &=  \frac{1}{1 + v_{WS}[W]}\label{eq:onewaySp}
\end{align}

\textbf{Reactions not in  Detailed Balance:}
This is the kinetic equation for reactions involving (de)phosphorylation. 
It is possible for either $S^u$ or $WS^u$ to undergo this process, so we have two equations
\begin{align}
\dv{[S^p]}{t} &=  k_{WS}^-[WS^p] - k_{WS}^+[W][S^p] + k_S^p[S^u] - k_S^u[S^p]\\
\dv{[WS^p]}{t} &= k_{WS}^+[W][S^p] -k_{WS}^-[WS^p] + k_{WS}^p[WS^u] + k_S^p[WS^u] - k_S^u[WS^p]
\end{align}
where we have defined the forward and backward dissociation rates $k_{WS}^+$ and $k_{WS}^-$ such that
\begin{align}
v_{WS} &= \frac{k_{WS}^+}{k_{WS}^-}
\end{align}
%Setting these to zero, we get
%\begin{align}
%0 &= k_{WS}^-[WS^p] - k_{WS}^+[W][S^p] + k_S^p[S^u] - k_S^u[S^p]\\
%0 &= k_{WS}^+[W][S^p] -k_{WS}^-[WS^p] + k_{WS}^p[WS^u] + k_S^p[WS^u] - k_S^u[WS^p]
%\end{align}
%

Adding these equations together, we eliminate these rates, giving us
\begin{align}
\dv{[S^p_T]}{t} &= k_{WS}^p[WS^u] + k_S^p[S_T^u] - k_S^u[S_T^p]\\
&= k_{WS}^p[S_T^u]p(WS^u|W) + k_S^p[S_T^u] - k_S^u[S_T^p]\\
&= k_{WS}^p([S_T]-[S_T^p])p(WS^u|W) + k_S^p([S_T]-[S_T^p]) - k_S^u[S_T^p]
\end{align}
Setting this to zero, we get
\begin{align}
\frac{[S_T^p]}{[S_T]} &= \frac{k_{WS}^p p(WS^u|W) + k_S^p}{k_{WS}^p p(WS^u|W) + k_S^p+k_S^u}\\
 &= \frac{1 + v_{WS}^p p(WS^u|W)}{1 + v_{WS}^p p(WS^u|W) + v_S^u}
\end{align}
where we have defined the reaction velocities relative to the background phosphorylation rate,
\begin{align}
v_{WS}^p &= \frac{k_{WS}^p}{k_S^p}\\
v_S^u &= \frac{k_S^u}{k_S^p}
\end{align}
If the total concentrations $[W_T]$ and $[S_T]$ are known, this equation plus Eqs.~\eqref{eq:onewayW}-\eqref{eq:onewaySu} and the four equations for concentration comprise a system of 8 equations for the remaining 7 unknown concentrations. Since this system is over-determined, either one of the concentrations or one of Eqs.~\eqref{eq:onewayW}-\eqref{eq:onewaySu} maybe neglected.


\section{Goldbeter Push-Pull Amplifier}

Intermediate states missing $WS^p$ and $ES^u$

\textbf{Reactions:}
\begin{gather}
W + S^u \leftrightarrows WS^u \rightarrow WS^p \rightarrow W + S^p\\
E + S^p \leftrightarrows ES^p \rightarrow ES^u \rightarrow E + S^u
\end{gather}
No background.

\section{Complete Push-Pull Amplifier}

\textbf{Reactions:}
\begin{gather}
W + S^u \leftrightarrows WS^u \rightarrow WS^p \leftrightarrows W + S^p\\
E + S^p \leftrightarrows ES^p \rightarrow ES^u \leftrightarrows E + S^u
\end{gather}

\textbf{Background:}
\begin{gather}
S^p \rightarrow S^u\\
S^u \rightarrow S^p\\
WS^p \rightarrow WS^u\\
WS^u \rightarrow WS^p\\
ES^p \rightarrow ES^u\\
ES^u \rightarrow ES^p
\end{gather}



\end{document}

