\pdfoutput=1
% ***********************************************************
% ******************* PHYSICS HEADER ************************
% ***********************************************************
% Version 2
%\documentclass[11pt,twocolumn]{article}
\documentclass[aps,onecolumn,superscriptaddress,notitlepage]{revtex4-1}


\usepackage{amsmath} % AMS Math Package
\usepackage{amsthm} % Theorem Formatting
\usepackage{amssymb}	% Math symbols such as \mathbb
\usepackage{physics}
\usepackage[pdftex]{graphicx}
\usepackage{hyperref} %clickable references.
\hypersetup{
    colorlinks,
    citecolor=blue,
    filecolor=black,
    linkcolor=red,
    urlcolor=blue
}
\usepackage{xcolor}
\usepackage{url}

\newcommand{\comment}[1]{{\color{blue}#1}}
\newcommand{\edit}[1]{{\color{violet}#1}}
\newcommand{\del}[1]{{\color{red}\st{#1}}}

\newcommand{\sigmaGFP}{\sigma_{\mathrm{GFP}}}
\newcommand{\sigmaanti}{\sigma_{\mathrm{anti}}}
\newcommand{\muGFP}{\mu_{\mathrm{GFP}}}
\newcommand{\muanti}{\mu_{\mathrm{anti}}}
\newcommand{\GFP}{[\mathrm{GFP}]}
\newcommand{\anti}{[\mathrm{anti}]}
\newcommand{\lGFP}{\log([\mathrm{GFP}])}
\newcommand{\lanti}{\log([\mathrm{anti}])}
\newcommand{\E}{\mathbf{E}}
\newcommand{\Var}{\mathrm{Var}}
\newcommand{\Cov}{\mathrm{Cov}}

% ***********************************************************
% ********************** END HEADER *************************
% ***********************************************************

\begin{document}

\title{Cheat Sheet for Push-Pull Models}
\maketitle

\section{Noise Model}

We assume the antibody and associated theoretical GFP measurements for the phosphorylated substrate each follow log-normal distributions. 
We define the means and variances as
\begin{align}
\E[\lanti] &= \muanti & \Var[\lGFP]&= \sigmaanti\\
\E[\lGFP] &= \muGFP & \Var[\lGFP] &= \sigmaGFP
\end{align}
so that the distributions are then given by
\begin{align}
P(\lanti) &= \frac{1}{\sqrt{2\pi\sigmaanti^2 }}\exp\qty(-\frac{(\lanti-\muanti)^2}{2\sigmaanti^2})\\
P(\lGFP) &= \frac{1}{\sqrt{2\pi\sigmaGFP^2 }}\exp\qty(-\frac{(\lGFP-\muGFP)^2}{2\sigmaGFP^2})
\end{align}

We define the Pearson correlations coefficient between the two measurements,
\begin{align}
\rho &= \frac{\Cov[\lGFP, \lanti]}{\sigmaGFP\sigmaanti},
\end{align}

The noise model for the phosphorylated substrate takes then takes the form of a conditional probability,
\begin{align}
P(\lanti | \lGFP) &= \frac{1}{\sqrt{\Sigma^2}}\exp\qty(-\frac{\qty[\lanti -A\lGFP - B]^2}{2 \Sigma^2})
\end{align}

\textbf{Fit Parameters:}
Above we used the the following definitions for the unknown noise parameters
\begin{itemize}
\item Conditional antibody variance:  $\Sigma^2 = \sigmaanti^2(1-\rho)$
\item GFP to antibody unit conversion ratio: $A =  \rho \frac{\sigmaanti}{\sigmaGFP}$
\item GFP to antobody unit constant offset: $B = \muanti - \rho \frac{\sigmaanti}{\sigmaGFP}\muGFP$
\end{itemize}


\section{Push Model}

\textbf{Fit Parameters:} $\alpha_{WS}$, $v_{WS}^p$, $v_{bg}^p$


\textbf{Concentrations:}
\begin{itemize}
\item Total writer: $[W_T] = [W] + [WS^u] + [WS^p]$
\item Total substrate: $[S_T] = [S^u_T] + [S^p_T]$
\item Total unphosporylated substrate: $[S^u_T] = [S^u] + [WS^u]$
\item Total phosporylated substrate: $[S^p_T] = [S^p] + [WS^p]$
\item Total unbound substrate: $[S_f] = [S^p] + [S^u]$
\end{itemize}

\textbf{Binding Energies:}
\begin{itemize}
\item Writer + substrate: $\Delta\epsilon_{WS}$
\end{itemize}

\textbf{Reaction Rates:}
\begin{itemize}
\item Phosphorylation of substrate by writer: $k_{WS}^p$
\item Background phosphorylation of substrate (independent of binding state): $k_{bg}^p$
\item Background dephosphorylation of substrate (independent of binding state): $k_{bg}^u$
\end{itemize}

\textbf{Fit Parameters}
\begin{itemize}
\item Phosphorylation velocity: $v_{WS}^p = k_{WS}^p/k_{bg}^u$
\item Background phosphorylation velocity  $v_{bg}^p = k_{WS}^p/k_{bg}^u$
\item Writer binding affinity: $\alpha_{WS} = [WS]_0e^{\beta \Delta\epsilon_{WS}}$
\end{itemize}
where $[WS]_0$ is a reference concentration typically defined as the concentration at half saturation.

\textbf{Model Equations}
The model then satisfies the following quadratic equation for the free substrate
\begin{align}
0 &= \qty(\frac{[S_f]}{\alpha_{WS}})^2 +  \qty(1+\frac{[W_T]-[S_T]}{\alpha_{WS}})\frac{[S_f]}{\alpha_{WS}} - \frac{[S_T]}{\alpha_{WS}}
\end{align}
After solving for $[S_f]$, the fraction of free writer is
\begin{align}
\frac{[W]}{[W_T]} & = \frac{1}{1 + \frac{[S_f]}{\alpha_{WS}}}
\end{align}
which can then be used to calculate the amount of phosphorylated substrate
\begin{align}
\frac{[S_T^p]}{[S_T]} &= \frac{v_{WS}^p p(WS^u|W) + v_S^p}{v_{WS}^p p(WS^u|W) + v_S^p+1}
\end{align}
where 
\begin{align}
p(WS^u|W) &= \frac{\frac{[W]}{\alpha_{WS}}}{1 + \frac{[W]}{\alpha_{WS}}}.
\end{align}


\section{Push-Pull Model}


\end{document}

